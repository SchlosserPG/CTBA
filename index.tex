% Options for packages loaded elsewhere
\PassOptionsToPackage{unicode}{hyperref}
\PassOptionsToPackage{hyphens}{url}
\PassOptionsToPackage{dvipsnames,svgnames,x11names}{xcolor}
%
\documentclass[
  letterpaper,
  DIV=11,
  numbers=noendperiod]{scrreprt}

\usepackage{amsmath,amssymb}
\usepackage{iftex}
\ifPDFTeX
  \usepackage[T1]{fontenc}
  \usepackage[utf8]{inputenc}
  \usepackage{textcomp} % provide euro and other symbols
\else % if luatex or xetex
  \usepackage{unicode-math}
  \defaultfontfeatures{Scale=MatchLowercase}
  \defaultfontfeatures[\rmfamily]{Ligatures=TeX,Scale=1}
\fi
\usepackage{lmodern}
\ifPDFTeX\else  
    % xetex/luatex font selection
\fi
% Use upquote if available, for straight quotes in verbatim environments
\IfFileExists{upquote.sty}{\usepackage{upquote}}{}
\IfFileExists{microtype.sty}{% use microtype if available
  \usepackage[]{microtype}
  \UseMicrotypeSet[protrusion]{basicmath} % disable protrusion for tt fonts
}{}
\makeatletter
\@ifundefined{KOMAClassName}{% if non-KOMA class
  \IfFileExists{parskip.sty}{%
    \usepackage{parskip}
  }{% else
    \setlength{\parindent}{0pt}
    \setlength{\parskip}{6pt plus 2pt minus 1pt}}
}{% if KOMA class
  \KOMAoptions{parskip=half}}
\makeatother
\usepackage{xcolor}
\setlength{\emergencystretch}{3em} % prevent overfull lines
\setcounter{secnumdepth}{5}
% Make \paragraph and \subparagraph free-standing
\makeatletter
\ifx\paragraph\undefined\else
  \let\oldparagraph\paragraph
  \renewcommand{\paragraph}{
    \@ifstar
      \xxxParagraphStar
      \xxxParagraphNoStar
  }
  \newcommand{\xxxParagraphStar}[1]{\oldparagraph*{#1}\mbox{}}
  \newcommand{\xxxParagraphNoStar}[1]{\oldparagraph{#1}\mbox{}}
\fi
\ifx\subparagraph\undefined\else
  \let\oldsubparagraph\subparagraph
  \renewcommand{\subparagraph}{
    \@ifstar
      \xxxSubParagraphStar
      \xxxSubParagraphNoStar
  }
  \newcommand{\xxxSubParagraphStar}[1]{\oldsubparagraph*{#1}\mbox{}}
  \newcommand{\xxxSubParagraphNoStar}[1]{\oldsubparagraph{#1}\mbox{}}
\fi
\makeatother

\usepackage{color}
\usepackage{fancyvrb}
\newcommand{\VerbBar}{|}
\newcommand{\VERB}{\Verb[commandchars=\\\{\}]}
\DefineVerbatimEnvironment{Highlighting}{Verbatim}{commandchars=\\\{\}}
% Add ',fontsize=\small' for more characters per line
\usepackage{framed}
\definecolor{shadecolor}{RGB}{241,243,245}
\newenvironment{Shaded}{\begin{snugshade}}{\end{snugshade}}
\newcommand{\AlertTok}[1]{\textcolor[rgb]{0.68,0.00,0.00}{#1}}
\newcommand{\AnnotationTok}[1]{\textcolor[rgb]{0.37,0.37,0.37}{#1}}
\newcommand{\AttributeTok}[1]{\textcolor[rgb]{0.40,0.45,0.13}{#1}}
\newcommand{\BaseNTok}[1]{\textcolor[rgb]{0.68,0.00,0.00}{#1}}
\newcommand{\BuiltInTok}[1]{\textcolor[rgb]{0.00,0.23,0.31}{#1}}
\newcommand{\CharTok}[1]{\textcolor[rgb]{0.13,0.47,0.30}{#1}}
\newcommand{\CommentTok}[1]{\textcolor[rgb]{0.37,0.37,0.37}{#1}}
\newcommand{\CommentVarTok}[1]{\textcolor[rgb]{0.37,0.37,0.37}{\textit{#1}}}
\newcommand{\ConstantTok}[1]{\textcolor[rgb]{0.56,0.35,0.01}{#1}}
\newcommand{\ControlFlowTok}[1]{\textcolor[rgb]{0.00,0.23,0.31}{\textbf{#1}}}
\newcommand{\DataTypeTok}[1]{\textcolor[rgb]{0.68,0.00,0.00}{#1}}
\newcommand{\DecValTok}[1]{\textcolor[rgb]{0.68,0.00,0.00}{#1}}
\newcommand{\DocumentationTok}[1]{\textcolor[rgb]{0.37,0.37,0.37}{\textit{#1}}}
\newcommand{\ErrorTok}[1]{\textcolor[rgb]{0.68,0.00,0.00}{#1}}
\newcommand{\ExtensionTok}[1]{\textcolor[rgb]{0.00,0.23,0.31}{#1}}
\newcommand{\FloatTok}[1]{\textcolor[rgb]{0.68,0.00,0.00}{#1}}
\newcommand{\FunctionTok}[1]{\textcolor[rgb]{0.28,0.35,0.67}{#1}}
\newcommand{\ImportTok}[1]{\textcolor[rgb]{0.00,0.46,0.62}{#1}}
\newcommand{\InformationTok}[1]{\textcolor[rgb]{0.37,0.37,0.37}{#1}}
\newcommand{\KeywordTok}[1]{\textcolor[rgb]{0.00,0.23,0.31}{\textbf{#1}}}
\newcommand{\NormalTok}[1]{\textcolor[rgb]{0.00,0.23,0.31}{#1}}
\newcommand{\OperatorTok}[1]{\textcolor[rgb]{0.37,0.37,0.37}{#1}}
\newcommand{\OtherTok}[1]{\textcolor[rgb]{0.00,0.23,0.31}{#1}}
\newcommand{\PreprocessorTok}[1]{\textcolor[rgb]{0.68,0.00,0.00}{#1}}
\newcommand{\RegionMarkerTok}[1]{\textcolor[rgb]{0.00,0.23,0.31}{#1}}
\newcommand{\SpecialCharTok}[1]{\textcolor[rgb]{0.37,0.37,0.37}{#1}}
\newcommand{\SpecialStringTok}[1]{\textcolor[rgb]{0.13,0.47,0.30}{#1}}
\newcommand{\StringTok}[1]{\textcolor[rgb]{0.13,0.47,0.30}{#1}}
\newcommand{\VariableTok}[1]{\textcolor[rgb]{0.07,0.07,0.07}{#1}}
\newcommand{\VerbatimStringTok}[1]{\textcolor[rgb]{0.13,0.47,0.30}{#1}}
\newcommand{\WarningTok}[1]{\textcolor[rgb]{0.37,0.37,0.37}{\textit{#1}}}

\providecommand{\tightlist}{%
  \setlength{\itemsep}{0pt}\setlength{\parskip}{0pt}}\usepackage{longtable,booktabs,array}
\usepackage{calc} % for calculating minipage widths
% Correct order of tables after \paragraph or \subparagraph
\usepackage{etoolbox}
\makeatletter
\patchcmd\longtable{\par}{\if@noskipsec\mbox{}\fi\par}{}{}
\makeatother
% Allow footnotes in longtable head/foot
\IfFileExists{footnotehyper.sty}{\usepackage{footnotehyper}}{\usepackage{footnote}}
\makesavenoteenv{longtable}
\usepackage{graphicx}
\makeatletter
\newsavebox\pandoc@box
\newcommand*\pandocbounded[1]{% scales image to fit in text height/width
  \sbox\pandoc@box{#1}%
  \Gscale@div\@tempa{\textheight}{\dimexpr\ht\pandoc@box+\dp\pandoc@box\relax}%
  \Gscale@div\@tempb{\linewidth}{\wd\pandoc@box}%
  \ifdim\@tempb\p@<\@tempa\p@\let\@tempa\@tempb\fi% select the smaller of both
  \ifdim\@tempa\p@<\p@\scalebox{\@tempa}{\usebox\pandoc@box}%
  \else\usebox{\pandoc@box}%
  \fi%
}
% Set default figure placement to htbp
\def\fps@figure{htbp}
\makeatother

\KOMAoption{captions}{tableheading}
\makeatletter
\@ifpackageloaded{bookmark}{}{\usepackage{bookmark}}
\makeatother
\makeatletter
\@ifpackageloaded{caption}{}{\usepackage{caption}}
\AtBeginDocument{%
\ifdefined\contentsname
  \renewcommand*\contentsname{Table of contents}
\else
  \newcommand\contentsname{Table of contents}
\fi
\ifdefined\listfigurename
  \renewcommand*\listfigurename{List of Figures}
\else
  \newcommand\listfigurename{List of Figures}
\fi
\ifdefined\listtablename
  \renewcommand*\listtablename{List of Tables}
\else
  \newcommand\listtablename{List of Tables}
\fi
\ifdefined\figurename
  \renewcommand*\figurename{Figure}
\else
  \newcommand\figurename{Figure}
\fi
\ifdefined\tablename
  \renewcommand*\tablename{Table}
\else
  \newcommand\tablename{Table}
\fi
}
\@ifpackageloaded{float}{}{\usepackage{float}}
\floatstyle{ruled}
\@ifundefined{c@chapter}{\newfloat{codelisting}{h}{lop}}{\newfloat{codelisting}{h}{lop}[chapter]}
\floatname{codelisting}{Listing}
\newcommand*\listoflistings{\listof{codelisting}{List of Listings}}
\makeatother
\makeatletter
\makeatother
\makeatletter
\@ifpackageloaded{caption}{}{\usepackage{caption}}
\@ifpackageloaded{subcaption}{}{\usepackage{subcaption}}
\makeatother

\usepackage{bookmark}

\IfFileExists{xurl.sty}{\usepackage{xurl}}{} % add URL line breaks if available
\urlstyle{same} % disable monospaced font for URLs
\hypersetup{
  pdftitle={CTBA},
  pdfauthor={Pamela Schlosser},
  colorlinks=true,
  linkcolor={blue},
  filecolor={Maroon},
  citecolor={Blue},
  urlcolor={Blue},
  pdfcreator={LaTeX via pandoc}}


\title{CTBA}
\author{Pamela Schlosser}
\date{2025-08-11}

\begin{document}
\maketitle

\renewcommand*\contentsname{Table of contents}
{
\hypersetup{linkcolor=}
\setcounter{tocdepth}{2}
\tableofcontents
}

\bookmarksetup{startatroot}

\chapter{How to Deploy a Dash App to Render.com (and Publish HTML via
docs/)}\label{how-to-deploy-a-dash-app-to-render.com-and-publish-html-via-docs}

\bookmarksetup{startatroot}

\chapter{Overview}\label{overview}

\begin{itemize}
\tightlist
\item
  This guide explains how to:

  \begin{itemize}
  \tightlist
  \item
    Prepare and run a Dash app locally
  \item
    Deploy it to Render.com using Gunicorn
  \item
    Fix common deployment issues
  \item
    Publish an HTML version of a Python file (or this guide) via GitHub
    Pages
  \item
    Use a docs/ folder to host HTML with GitHub Pages
  \end{itemize}
\end{itemize}

\bookmarksetup{startatroot}

\chapter{Live Example (Embedded
Iframe)}\label{live-example-embedded-iframe}

\begin{itemize}
\tightlist
\item
  This shows a hosted Dash app inside the HTML output of this page.
\item
  Replace the src URL with your own Render app link if different.
\end{itemize}

\begin{Shaded}
\begin{Highlighting}[]
\NormalTok{\textless{}iframe src="https://ctba{-}oror.onrender.com/" title="Live Dash App on Render" style="width:100\%;height:500px;border:2px solid \#115740;border{-}radius:6px;"\textgreater{}\textless{}/iframe\textgreater{}}
\end{Highlighting}
\end{Shaded}

\begin{enumerate}
\def\labelenumi{\arabic{enumi})}
\tightlist
\item
  Prepare Your Dash App
\end{enumerate}

\begin{itemize}
\item
  Ensure your main Python file defines: server = app.server
\item
  The Gunicorn start command on Render must match your filename and
  server variable exactly
\end{itemize}

\begin{Shaded}
\begin{Highlighting}[]
\ImportTok{from}\NormalTok{ dash }\ImportTok{import}\NormalTok{ Dash, html}

\NormalTok{app }\OperatorTok{=}\NormalTok{ Dash(}\VariableTok{\_\_name\_\_}\NormalTok{)}
\NormalTok{server }\OperatorTok{=}\NormalTok{ app.server  }\CommentTok{\# Required for Gunicorn in production}

\NormalTok{wm\_green }\OperatorTok{=} \StringTok{"\#115740"}

\NormalTok{app.layout }\OperatorTok{=}\NormalTok{ html.Div([}
\NormalTok{    html.H1(}\StringTok{"Hello from Dash on Render!"}\NormalTok{, style}\OperatorTok{=}\NormalTok{\{}\StringTok{"color"}\NormalTok{: wm\_green, }\StringTok{"textAlign"}\NormalTok{: }\StringTok{"center"}\NormalTok{\})}
\NormalTok{])}

\ControlFlowTok{if} \VariableTok{\_\_name\_\_} \OperatorTok{==} \StringTok{"\_\_main\_\_"}\NormalTok{:}
\NormalTok{    app.run(debug}\OperatorTok{=}\VariableTok{False}\NormalTok{)}
\end{Highlighting}
\end{Shaded}

\begin{verbatim}
<IPython.lib.display.IFrame at 0x23928fadfd0>
\end{verbatim}

\begin{itemize}
\tightlist
\item
  Example mapping:

  \begin{itemize}
  \tightlist
  \item
    File: LiveDash.py --\textgreater{} Start command: gunicorn
    LiveDash:server
  \item
    File: electricity.py --\textgreater{} Start command: gunicorn
    electricity:server
  \end{itemize}
\end{itemize}

\begin{enumerate}
\def\labelenumi{\arabic{enumi})}
\setcounter{enumi}{1}
\tightlist
\item
  Add Required Files
\end{enumerate}

\begin{itemize}
\tightlist
\item
  Create a requirements.txt listing dependencies (add any others you
  import).
\end{itemize}

\begin{Shaded}
\begin{Highlighting}[]
\NormalTok{dash}
\NormalTok{gunicorn}
\NormalTok{plotly}
\end{Highlighting}
\end{Shaded}

\begin{enumerate}
\def\labelenumi{\arabic{enumi})}
\setcounter{enumi}{2}
\tightlist
\item
  Run Locally
\end{enumerate}

\begin{itemize}
\tightlist
\item
  Create and activate a virtual environment
\item
  Install dependencies
\item
  Test both dev mode and production mode
\end{itemize}

'\,'\,'bash python -m venv .venv \# Windows PowerShell:
.venv\Scripts\Activate.ps1 \# macOS/Linux: source .venv/bin/activate

pip install -r requirements.txt python your\_filename.py \# dev mode
gunicorn your\_filename:server \# production test '\,'\,'

\begin{enumerate}
\def\labelenumi{\arabic{enumi})}
\setcounter{enumi}{3}
\tightlist
\item
  Push to GitHub
\end{enumerate}

\begin{Shaded}
\begin{Highlighting}[]
\NormalTok{git init}
\NormalTok{git add .}
\NormalTok{git commit {-}m "Initial commit"}
\NormalTok{git branch {-}M main}
\NormalTok{git remote add origin https://github.com/USERNAME/REPO.git}
\NormalTok{git push {-}u origin main}
\end{Highlighting}
\end{Shaded}

\begin{enumerate}
\def\labelenumi{\arabic{enumi})}
\setcounter{enumi}{4}
\tightlist
\item
  Deploy on Render.com
\end{enumerate}

\begin{itemize}
\tightlist
\item
  Create a New Web Service on Render and connect your GitHub repo
\item
  Build Command: pip install -r requirements.txt
\item
  Start Command: gunicorn your\_filename:server
\item
  Click Deploy and open the provided URL
\end{itemize}

\begin{enumerate}
\def\labelenumi{\arabic{enumi})}
\setcounter{enumi}{5}
\tightlist
\item
  Troubleshooting
\end{enumerate}

\begin{itemize}
\tightlist
\item
  ModuleNotFoundError: Check filename and server variable
  (your\_filename:server)
\item
  Changes not showing: Clear build cache and redeploy
\item
  Missing packages: Confirm all imports are in requirements.txt
\item
  Port errors: Don't hardcode PORT---Render sets it automatically
\end{itemize}

\begin{enumerate}
\def\labelenumi{\arabic{enumi})}
\setcounter{enumi}{6}
\tightlist
\item
  Publish HTML via docs/ on GitHub Pages
\end{enumerate}

\begin{itemize}
\tightlist
\item
  Render this Quarto file to docs/ for GitHub Pages:
\end{itemize}

\begin{Shaded}
\begin{Highlighting}[]
\NormalTok{quarto render dash\_to\_render.qmd {-}{-}to html {-}{-}output{-}dir docs }
\NormalTok{git add docs }
\NormalTok{git commit {-}m "Publish HTML" }
\NormalTok{git push }
\end{Highlighting}
\end{Shaded}

GitHub Pages setup:

\begin{itemize}
\tightlist
\item
  Settings → Pages
\item
  Source: Deploy from a branch
\item
  Branch: main
\item
  Folder: /docs
\item
  Save
\end{itemize}

\bookmarksetup{startatroot}

\chapter{Your site will be available
at:}\label{your-site-will-be-available-at}

\begin{itemize}
\tightlist
\item
  https://YOUR-USERNAME.github.io/YOUR-REPO/
\end{itemize}

\section{Appendix A: Render a Python file to HTML with
Quarto}\label{appendix-a-render-a-python-file-to-html-with-quarto}

Quarto can render a Python script (with outputs) to HTML.

\begin{Shaded}
\begin{Highlighting}[]
\NormalTok{quarto render your\_script.py {-}{-}to html {-}{-}output{-}dir docs}
\end{Highlighting}
\end{Shaded}





\end{document}
